    \documentclass[12pt,a4paper]{article}

    % Paquetes necesarios
    \usepackage[utf8]{inputenc}
    \usepackage[spanish]{babel}
    \usepackage{graphicx}
    \usepackage{geometry}
    \usepackage{fancyhdr}
    \usepackage{titlesec}
    \usepackage{enumitem}
    \usepackage{hyperref}
    \usepackage{xcolor}
    \usepackage{listings}
    \usepackage{tabularx}
    \usepackage{multirow}
    \usepackage{booktabs}
    \usepackage{float}
    \usepackage{longtable}
    \usepackage{pdflscape}
    \usepackage[backend=biber,style=ieee,sorting=none]{biblatex}

    % Configuración de márgenes
    \geometry{top=2.5cm, bottom=2.5cm, left=3cm, right=2.5cm}

    % Configuración de encabezados y pies de página
    \pagestyle{fancy}
    \fancyhf{}
    \fancyhead[L]{\small Diseño de Pruebas - TravelBrain}
    \fancyhead[R]{\small ESPE 2026}
    \fancyfoot[C]{\thepage}
    \renewcommand{\headrulewidth}{0.4pt}
    \renewcommand{\footrulewidth}{0.4pt}

    % Configuración de hipervínculos
    \hypersetup{
        colorlinks=true,
        linkcolor=blue,
        filecolor=magenta,      
        urlcolor=cyan,
        citecolor=blue,
        pdftitle={Diseño de Pruebas - TravelBrain},
        pdfauthor={Cáceres Germán, Anthony Villareal},
    }

    % Configuración de colores para código
    \definecolor{codegreen}{rgb}{0,0.6,0}
    \definecolor{codegray}{rgb}{0.5,0.5,0.5}
    \definecolor{codepurple}{rgb}{0.58,0,0.82}
    \definecolor{backcolour}{rgb}{0.95,0.95,0.92}

    % Definir lenguaje JavaScript para listings
    \lstdefinelanguage{JavaScript}{
    keywords={typeof, new, true, false, catch, function, return, null, catch, switch, var, if, in, while, do, else, case, break, const, let, async, await, class, extends, export, import, require},
    keywordstyle=\color{blue}\bfseries,
    ndkeywords={class, export, boolean, throw, implements, import, this},
    ndkeywordstyle=\color{darkgray}\bfseries,
    identifierstyle=\color{black},
    sensitive=false,
    comment=[l]{//},
    morecomment=[s]{/*}{*/},
    commentstyle=\color{codegreen}\ttfamily,
    stringstyle=\color{codepurple}\ttfamily,
    morestring=[b]",
    morestring=[b]"
    }

    \lstdefinestyle{mystyle}{
        backgroundcolor=\color{backcolour},   
        commentstyle=\color{codegreen},
        keywordstyle=\color{magenta},
        numberstyle=\tiny\color{codegray},
        stringstyle=\color{codepurple},
        basicstyle=\ttfamily\footnotesize,
        breakatwhitespace=false,         
        breaklines=true,                 
        captionpos=b,                    
        keepspaces=true,                 
        numbers=left,                    
        numbersep=5pt,                  
        showspaces=false,                
        showstringspaces=false,
        showtabs=false,                  
        tabsize=2,
        literate=
            {á}{{\'a}}1 {é}{{\'e}}1 {í}{{\'i}}1 {ó}{{\'o}}1 {ú}{{\'u}}1
            {Á}{{\'A}}1 {É}{{\'E}}1 {Í}{{\'I}}1 {Ó}{{\'O}}1 {Ú}{{\'U}}1
            {ñ}{{\~n}}1 {Ñ}{{\~N}}1
    }

    \lstset{style=mystyle}

    % Archivo de bibliografía
    \addbibresource{referencias.bib}

    \begin{document}

    % ============================================================
    % PORTADA
    % ============================================================
    \begin{titlepage}
        \centering
        \vspace*{1cm}
        
        {\Large \textbf{UNIVERSIDAD DE LAS FUERZAS ARMADAS ESPE}} \\[0.5cm]
        {\large Departamento de Ciencias de la Computación} \\[1.5cm]
        
        \includegraphics[width=0.3\textwidth]{img/ESPE.png} \\[1cm]
        
        {\LARGE \textbf{Proyecto Final: Plan de Aseguramiento de la Calidad (SQAP)}} \\[0.5cm]
        {\Large \textbf{Sistema TravelBrain}} \\[1.5cm]
        
        {\large \textbf{Informe 2: Diseño de Casos de Prueba y Matrices de Rastreabilidad}} \\[2cm]
        
        \begin{tabular}{ll}
            \textbf{Asignatura:} & Aseguramiento de la Calidad del Software \\
            \textbf{NRC:} & 27886 \\
            \textbf{Estudiantes:} & Cáceres Germán (Scrum Master \& Tech Lead) \\
                                & Anthony Villareal (Development Team) \\
            \textbf{Docente:} & Ing. Diego Gamboa, Mgs. \\
            \textbf{Fecha:} & 21 de enero de 2026 \\
        \end{tabular}
        
        \vfill
        
        {\large Sangolquí, Ecuador} \\
        {\large 2026}
    \end{titlepage}

    % ============================================================
    % ÍNDICE
    % ============================================================
    \tableofcontents
    \newpage

    % ============================================================
    % LISTA DE TABLAS
    % ============================================================
    \listoftables
    \newpage

    % ============================================================
    % RESUMEN EJECUTIVO
    % ============================================================
    \section*{Resumen Ejecutivo}
    \addcontentsline{toc}{section}{Resumen Ejecutivo}

    El presente documento constituye el \textbf{Diseño Detallado de Casos de Prueba} para el sistema TravelBrain, complementando el Plan Maestro de Pruebas establecido en el Informe 1.

    Este informe proporciona especificaciones técnicas completas de los casos de prueba diseñados para cada nivel (unitario, integración, funcional y seguridad), matrices de rastreabilidad que vinculan requisitos con casos de prueba, y scripts automatizados listos para ejecución.

    Aunque las pruebas \textbf{aún no han sido ejecutadas}, este documento establece la base metodológica para la fase de ejecución del Sprint de Calidad, asegurando cobertura exhaustiva de funcionalidades críticas y trazabilidad completa desde requisitos hasta casos de prueba.

    \textbf{Total de casos diseñados:} 87 casos de prueba distribuidos en 4 niveles.

    \textbf{Palabras clave:} Casos de Prueba, Matriz de Rastreabilidad, Test Design, Cypress, Jest, Postman, OWASP ZAP.

    \newpage

    % ============================================================
    % 1. INTRODUCCIÓN
    % ============================================================
    \section{Introducción}

    \subsection{Propósito del Documento}

    Este documento detalla el diseño completo de casos de prueba para el sistema TravelBrain, proporcionando especificaciones técnicas que servirán como guía durante la fase de ejecución del Sprint de Calidad.

    \subsection{Alcance del Diseño}

    El diseño cubre los siguientes niveles de prueba:

    \begin{enumerate}
        \item \textbf{Pruebas Unitarias:} Componentes individuales (Frontend + Backend)
        \item \textbf{Pruebas de Integración:} APIs RESTful y comunicación entre servicios
        \item \textbf{Pruebas Funcionales E2E:} Flujos completos de usuario
        \item \textbf{Pruebas de Seguridad:} Vulnerabilidades OWASP Top 10
    \end{enumerate}

    \subsection{Referencias}

    Este documento se basa en:

    \begin{itemize}
        \item \textbf{Informe 1:} Plan Maestro de Pruebas (SQAP)
        \item \textbf{ARCHITECTURE.md:} Documentación técnica del sistema
        \item \textbf{IEEE 829:} Estándar para documentación de pruebas \cite{ieee829}
        \item \textbf{ISO/IEC/IEEE 29119:} Estándar de pruebas de software \cite{iso29119}
    \end{itemize}

    % ============================================================
    % 2. REQUISITOS FUNCIONALES
    % ============================================================
    \section{Requisitos Funcionales del Sistema}

    \subsection{Módulo de Autenticación}

    \begin{table}[H]
    \centering
    \caption{Requisitos Funcionales - Autenticación}
    \label{tab:req_autenticacion}
    \small
    \begin{tabularx}{\textwidth}{|l|X|c|}
    \hline
    \textbf{ID} & \textbf{Requisito} & \textbf{Prioridad} \\ \hline
    RF-AUTH-01 & El sistema debe permitir registro de usuarios con email, username y password & Alta \\ \hline
    RF-AUTH-02 & El sistema debe validar formato de email (RFC 5322) & Alta \\ \hline
    RF-AUTH-03 & El sistema debe requerir contraseñas con mínimo 8 caracteres & Alta \\ \hline
    RF-AUTH-04 & El sistema debe hashear contraseñas con bcrypt antes de almacenar & Alta \\ \hline
    RF-AUTH-05 & El sistema debe permitir login con email y password & Alta \\ \hline
    RF-AUTH-06 & El sistema debe generar token JWT con expiración de 24h tras login exitoso & Alta \\ \hline
    RF-AUTH-07 & El sistema debe rechazar credenciales inválidas con mensaje apropiado & Alta \\ \hline
    RF-AUTH-08 & El sistema debe permitir logout y invalidar token JWT & Media \\ \hline
    \end{tabularx}
    \end{table}

    \subsection{Módulo de Autenticación Biométrica}

    \begin{table}[H]
    \centering
    \caption{Requisitos Funcionales - Biometría}
    \label{tab:req_biometria}
    \small
    \begin{tabularx}{\textwidth}{|l|X|c|}
    \hline
    \textbf{ID} & \textbf{Requisito} & \textbf{Prioridad} \\ \hline
    RF-BIO-01 & El sistema debe permitir registro de datos biométricos faciales & Alta \\ \hline
    RF-BIO-02 & El sistema debe generar challenge token temporal (TTL 2 min) para verificación & Alta \\ \hline
    RF-BIO-03 & El sistema debe extraer encoding 128D mediante microservicio facial & Alta \\ \hline
    RF-BIO-04 & El sistema debe cifrar encodings con AES-256-CBC antes de almacenar & Alta \\ \hline
    RF-BIO-05 & El sistema debe validar liveness score $\geq$ 0.6 & Alta \\ \hline
    RF-BIO-06 & El sistema debe validar quality score $\geq$ 0.6 & Alta \\ \hline
    RF-BIO-07 & El sistema debe comparar encodings con threshold 0.6 para autenticación & Alta \\ \hline
    RF-BIO-08 & El sistema debe registrar auditoría de intentos biométricos & Media \\ \hline
    \end{tabularx}
    \end{table}

    \subsection{Módulo de Gestión de Viajes}

    \begin{table}[H]
    \centering
    \caption{Requisitos Funcionales - Viajes}
    \label{tab:req_viajes}
    \small
    \begin{tabularx}{\textwidth}{|l|X|c|}
    \hline
    \textbf{ID} & \textbf{Requisito} & \textbf{Prioridad} \\ \hline
    RF-TRIP-01 & El sistema debe permitir crear viajes con título, destino, fechas y presupuesto & Alta \\ \hline
    RF-TRIP-02 & El sistema debe validar que startDate $<$ endDate & Alta \\ \hline
    RF-TRIP-03 & El sistema debe permitir listar viajes del usuario autenticado & Alta \\ \hline
    RF-TRIP-04 & El sistema debe permitir actualizar datos de un viaje propio & Media \\ \hline
    RF-TRIP-05 & El sistema debe permitir eliminar un viaje propio & Media \\ \hline
    RF-TRIP-06 & El sistema debe impedir acceso a viajes de otros usuarios (IDOR protection) & Alta \\ \hline
    \end{tabularx}
    \end{table}

    \subsection{Módulo de Clima}

    \begin{table}[H]
    \centering
    \caption{Requisitos Funcionales - Clima}
    \label{tab:req_clima}
    \small
    \begin{tabularx}{\textwidth}{|l|X|c|}
    \hline
    \textbf{ID} & \textbf{Requisito} & \textbf{Prioridad} \\ \hline
    RF-WEAT-01 & El sistema debe buscar clima actual mediante OpenWeather API & Media \\ \hline
    RF-WEAT-02 & El sistema debe almacenar historial de búsquedas de clima & Media \\ \hline
    RF-WEAT-03 & El sistema debe mostrar temperatura, condiciones e ícono & Media \\ \hline
    RF-WEAT-04 & El sistema debe permitir listar historial de búsquedas & Baja \\ \hline
    \end{tabularx}
    \end{table}

    \subsection{Módulo de Administración}

    \begin{table}[H]
    \centering
    \caption{Requisitos Funcionales - Admin}
    \label{tab:req_admin}
    \small
    \begin{tabularx}{\textwidth}{|l|X|c|}
    \hline
    \textbf{ID} & \textbf{Requisito} & \textbf{Prioridad} \\ \hline
    RF-ADM-01 & El sistema debe permitir a ADMIN listar todos los usuarios & Alta \\ \hline
    RF-ADM-02 & El sistema debe permitir a ADMIN cambiar roles de usuarios & Alta \\ \hline
    RF-ADM-03 & El sistema debe impedir a usuarios regulares acceder a rutas admin & Alta \\ \hline
    \end{tabularx}
    \end{table}

    % ============================================================
    % 3. MATRIZ DE RASTREABILIDAD
    % ============================================================
    \section{Matriz de Rastreabilidad: Requisitos vs Casos de Prueba}

    \begin{landscape}
    \begin{longtable}{|l|l|p{6cm}|l|l|}
    \caption{Matriz de Rastreabilidad - Requisitos vs Casos de Prueba} \label{tab:rastreabilidad} \\
    \hline
    \textbf{ID Req.} & \textbf{ID Caso} & \textbf{Nombre del Caso de Prueba} & \textbf{Nivel} & \textbf{Herramienta} \\ \hline
    \endfirsthead

    \multicolumn{5}{c}{\tablename\ \thetable\ -- Continuación} \\
    \hline
    \textbf{ID Req.} & \textbf{ID Caso} & \textbf{Nombre del Caso de Prueba} & \textbf{Nivel} & \textbf{Herramienta} \\ \hline
    \endhead

    \hline \multicolumn{5}{r}{Continúa en la siguiente página...} \\
    \endfoot

    \hline
    \endlastfoot

    % AUTENTICACIÓN
    RF-AUTH-01 & TC-AUTH-001 & Registro exitoso con datos válidos & E2E & Cypress \\ \hline
    RF-AUTH-01 & TC-AUTH-002 & Registro falla con email duplicado & E2E & Cypress \\ \hline
    RF-AUTH-02 & TC-AUTH-003 & Registro rechaza email inválido & Integración & Postman \\ \hline
    RF-AUTH-03 & TC-AUTH-004 & Registro rechaza password corto & Integración & Postman \\ \hline
    RF-AUTH-04 & TC-AUTH-U01 & bcrypt hashea contraseñas correctamente & Unitaria & Jest \\ \hline
    RF-AUTH-05 & TC-AUTH-005 & Login exitoso con credenciales válidas & E2E & Cypress \\ \hline
    RF-AUTH-06 & TC-AUTH-U02 & JWT generado contiene userId y expira en 24h & Unitaria & Jest \\ \hline
    RF-AUTH-07 & TC-AUTH-006 & Login falla con credenciales inválidas & E2E & Cypress \\ \hline
    RF-AUTH-08 & TC-AUTH-007 & Logout exitoso elimina token del cliente & E2E & Cypress \\ \hline

    % BIOMETRÍA
    RF-BIO-01 & TC-BIO-001 & Registro biométrico exitoso con imagen válida & E2E & Cypress \\ \hline
    RF-BIO-02 & TC-BIO-002 & Challenge token generado expira en 2 minutos & Integración & Postman \\ \hline
    RF-BIO-03 & TC-BIO-003 & Microservicio extrae encoding 128D correctamente & Integración & Postman \\ \hline
    RF-BIO-04 & TC-BIO-U01 & Encoding cifrado con AES-256 puede descifrarse & Unitaria & Jest \\ \hline
    RF-BIO-05 & TC-BIO-004 & Verificación rechaza imagen con liveness bajo & Integración & Postman \\ \hline
    RF-BIO-06 & TC-BIO-005 & Verificación rechaza imagen de baja calidad & Integración & Postman \\ \hline
    RF-BIO-07 & TC-BIO-006 & Login biométrico exitoso con rostro registrado & E2E & Cypress \\ \hline
    RF-BIO-08 & TC-BIO-U02 & Auditoría registra timestamp e IP del intento & Unitaria & Jest \\ \hline

    % VIAJES
    RF-TRIP-01 & TC-TRIP-001 & Crear viaje exitosamente con datos completos & E2E & Cypress \\ \hline
    RF-TRIP-02 & TC-TRIP-002 & Creación rechaza startDate posterior a endDate & Integración & Postman \\ \hline
    RF-TRIP-03 & TC-TRIP-003 & Listar viajes retorna solo viajes propios & Integración & Postman \\ \hline
    RF-TRIP-04 & TC-TRIP-004 & Actualizar viaje propio exitosamente & E2E & Cypress \\ \hline
    RF-TRIP-05 & TC-TRIP-005 & Eliminar viaje propio exitosamente & E2E & Cypress \\ \hline
    RF-TRIP-06 & TC-TRIP-006 & Acceso a viaje ajeno retorna 403 Forbidden & Integración & Postman \\ \hline
    RF-TRIP-06 & TC-SEC-001 & Prevención de IDOR en endpoint /trips/:id & Seguridad & OWASP ZAP \\ \hline

    % CLIMA
    RF-WEAT-01 & TC-WEAT-001 & Búsqueda de clima retorna datos válidos & Integración & Postman \\ \hline
    RF-WEAT-02 & TC-WEAT-002 & Búsqueda almacena registro en BD & Unitaria & Jest \\ \hline
    RF-WEAT-03 & TC-WEAT-003 & Respuesta incluye temp, description e icon & Integración & Postman \\ \hline
    RF-WEAT-04 & TC-WEAT-004 & Listar historial retorna búsquedas ordenadas & E2E & Cypress \\ \hline

    % ADMINISTRACIÓN
    RF-ADM-01 & TC-ADM-001 & Admin lista todos los usuarios & Integración & Postman \\ \hline
    RF-ADM-02 & TC-ADM-002 & Admin cambia rol de usuario exitosamente & Integración & Postman \\ \hline
    RF-ADM-03 & TC-ADM-003 & Usuario regular no accede a /api/users & Integración & Postman \\ \hline
    RF-ADM-03 & TC-SEC-002 & Protección de rutas admin sin bypass & Seguridad & OWASP ZAP \\ \hline

    % SEGURIDAD GENERAL
    - & TC-SEC-003 & Detección de vulnerabilidades XSS en formularios & Seguridad & OWASP ZAP \\ \hline
    - & TC-SEC-004 & Detección de SQL Injection en parámetros & Seguridad & OWASP ZAP \\ \hline
    - & TC-SEC-005 & Headers de seguridad presentes (CSP, HSTS) & Seguridad & OWASP ZAP \\ \hline
    - & TC-SEC-006 & Sin exposición de información sensible en errores & Seguridad & OWASP ZAP \\ \hline
    - & TC-SEC-007 & JWT no puede ser manipulado sin invalidar firma & Seguridad & Manual \\ \hline

    \end{longtable}
    \end{landscape}

    % ============================================================
    % 4. DISEÑO DE CASOS DE PRUEBA UNITARIAS
    % ============================================================
    \section{Diseño de Casos de Prueba Unitarias}

    \subsection{Casos de Prueba Unitaria - Backend (Jest)}

    \subsubsection{TC-AUTH-U01: Hashing de Contraseñas con bcrypt}

    \begin{table}[H]
    \centering
    \caption{TC-AUTH-U01: Hashing de Contraseñas}
    \label{tab:tc_auth_u01}
    \begin{tabularx}{\textwidth}{|l|X|}
    \hline
    \textbf{ID} & TC-AUTH-U01 \\ \hline
    \textbf{Nombre} & Verificar hashing correcto de contraseñas con bcrypt \\ \hline
    \textbf{Módulo} & authController.js - register() \\ \hline
    \textbf{Prioridad} & Alta \\ \hline
    \textbf{Precondiciones} & 
    \begin{itemize}[leftmargin=*, nosep, after=\vspace{-\baselineskip}]
        \item Módulo bcrypt importado
        \item Función register() disponible
    \end{itemize} \\ \hline
    \textbf{Datos de Entrada} & 
    \begin{itemize}[leftmargin=*, nosep, after=\vspace{-\baselineskip}]
        \item password: "Test123!"
    \end{itemize} \\ \hline
    \textbf{Pasos} & 
    \begin{enumerate}[leftmargin=*, nosep, after=\vspace{-\baselineskip}]
        \item Mock de User.findOne() retornando null
        \item Mock de bcrypt.hash() retornando hash predeterminado
        \item Llamar a register() con datos de prueba
        \item Verificar que bcrypt.hash fue llamado con password
    \end{enumerate} \\ \hline
    \textbf{Resultado Esperado} & 
    \begin{itemize}[leftmargin=*, nosep, after=\vspace{-\baselineskip}]
        \item bcrypt.hash() llamado exactamente 1 vez
        \item Password no almacenado en texto plano
        \item Hash almacenado en BD es string de 60 caracteres
    \end{itemize} \\ \hline
    \textbf{Resultado Obtenido} & \textit{Pendiente de ejecución} \\ \hline
    \textbf{Estado} & No Ejecutado \\ \hline
    \end{tabularx}
    \end{table}

    \textbf{Script de Prueba:}

    \begin{lstlisting}[language=JavaScript, caption={TC-AUTH-U01: Jest Script}]
    const { register } = require("../src/controllers/authController");
    const User = require("../src/models/User");
    const bcrypt = require("bcrypt");

    jest.mock("../src/models/User");
    jest.mock("bcrypt");

    describe("TC-AUTH-U01: Password Hashing with bcrypt", () => {
    let req, res;

    beforeEach(() => {
        req = {
        body: {
            email: "test@mail.com",
            password: "Test123!",
            username: "testuser"
        }
        };
        res = {
        status: jest.fn().mockReturnThis(),
        json: jest.fn()
        };
    });

    afterEach(() => {
        jest.clearAllMocks();
    });

    it("should hash password with bcrypt before storing", async () => {
        // Arrange
        const hashedPassword = "$2b$10$abcdefghijklmnopqrstuvwxyz123456";
        User.findOne.mockResolvedValue(null);
        bcrypt.hash.mockResolvedValue(hashedPassword);
        User.prototype.save = jest.fn().mockResolvedValue({
        _id: "123",
        email: "test@mail.com",
        password: hashedPassword
        });

        // Act
        await register(req, res);

        // Assert
        expect(bcrypt.hash).toHaveBeenCalledTimes(1);
        expect(bcrypt.hash).toHaveBeenCalledWith("Test123!", 10);
        expect(User.prototype.save).toHaveBeenCalled();
        const savedUser = await User.prototype.save.mock.results[0].value;
        expect(savedUser.password).toBe(hashedPassword);
        expect(savedUser.password).toHaveLength(60);
    });

    it("should not store password in plain text", async () => {
        User.findOne.mockResolvedValue(null);
        bcrypt.hash.mockResolvedValue("hashedPassword");
        User.prototype.save = jest.fn().mockResolvedValue({
        password: "hashedPassword"
        });

        await register(req, res);

        const savedUser = await User.prototype.save.mock.results[0].value;
        expect(savedUser.password).not.toBe("Test123!");
    });
    });
    \end{lstlisting}

    \subsubsection{TC-AUTH-U02: Generación y Expiración de JWT}

    \begin{table}[H]
    \centering
    \caption{TC-AUTH-U02: Generación de JWT}
    \label{tab:tc_auth_u02}
    \begin{tabularx}{\textwidth}{|l|X|}
    \hline
    \textbf{ID} & TC-AUTH-U02 \\ \hline
    \textbf{Nombre} & Verificar generación correcta de JWT con userId y expiración 24h \\ \hline
    \textbf{Módulo} & authController.js - login() \\ \hline
    \textbf{Prioridad} & Alta \\ \hline
    \textbf{Precondiciones} & 
    \begin{itemize}[leftmargin=*, nosep, after=\vspace{-\baselineskip}]
        \item Módulo jsonwebtoken importado
        \item JWT\_SECRET configurado
    \end{itemize} \\ \hline
    \textbf{Datos de Entrada} & 
    \begin{itemize}[leftmargin=*, nosep, after=\vspace{-\baselineskip}]
        \item userId: "507f1f77bcf86cd799439011"
    \end{itemize} \\ \hline
    \textbf{Pasos} & 
    \begin{enumerate}[leftmargin=*, nosep, after=\vspace{-\baselineskip}]
        \item Mock de User.findOne() retornando usuario válido
        \item Mock de bcrypt.compare() retornando true
        \item Llamar a login() con credenciales válidas
        \item Decodificar JWT retornado
        \item Verificar payload y expiración
    \end{enumerate} \\ \hline
    \textbf{Resultado Esperado} & 
    \begin{itemize}[leftmargin=*, nosep, after=\vspace{-\baselineskip}]
        \item JWT contiene userId en payload
        \item exp (expiration) = iat + 24 horas (86400 seg)
        \item JWT firma válida con JWT\_SECRET
    \end{itemize} \\ \hline
    \textbf{Resultado Obtenido} & \textit{Pendiente de ejecución} \\ \hline
    \textbf{Estado} & No Ejecutado \\ \hline
    \end{tabularx}
    \end{table}

    \textbf{Script de Prueba:}

    \begin{lstlisting}[language=JavaScript, caption={TC-AUTH-U02: Jest Script}]
    const { login } = require("../src/controllers/authController");
    const User = require("../src/models/User");
    const bcrypt = require("bcrypt");
    const jwt = require("jsonwebtoken");

    jest.mock("../src/models/User");
    jest.mock("bcrypt");

    describe("TC-AUTH-U02: JWT Generation and Expiration", () => {
    const JWT_SECRET = "test-secret-key";
    process.env.JWT_SECRET = JWT_SECRET;

    let req, res;

    beforeEach(() => {
        req = {
        body: {
            email: "test@mail.com",
            password: "Test123!"
        }
        };
        res = {
        status: jest.fn().mockReturnThis(),
        json: jest.fn()
        };
    });

    it("should generate JWT with userId and 24h expiration", async () => {
        // Arrange
        const mockUser = {
        _id: "507f1f77bcf86cd799439011",
        email: "test@mail.com",
        password: "hashedPassword",
        username: "testuser"
        };
        User.findOne.mockResolvedValue(mockUser);
        bcrypt.compare.mockResolvedValue(true);

        // Act
        await login(req, res);

        // Assert
        expect(res.status).toHaveBeenCalledWith(200);
        const response = res.json.mock.calls[0][0];
        const token = response.data.token;

        // Decode JWT
        const decoded = jwt.verify(token, JWT_SECRET);
        
        expect(decoded.userId).toBe(mockUser._id);
        
        // Verify 24h expiration (86400 seconds)
        const expirationTime = decoded.exp - decoded.iat;
        expect(expirationTime).toBe(86400);
    });

    it("should sign JWT with JWT_SECRET", async () => {
        const mockUser = {
        _id: "123",
        email: "test@mail.com",
        password: "hashedPassword"
        };
        User.findOne.mockResolvedValue(mockUser);
        bcrypt.compare.mockResolvedValue(true);

        await login(req, res);

        const response = res.json.mock.calls[0][0];
        const token = response.data.token;

        // Should not throw error
        expect(() => {
        jwt.verify(token, JWT_SECRET);
        }).not.toThrow();

        // Should throw with wrong secret
        expect(() => {
        jwt.verify(token, "wrong-secret");
        }).toThrow();
    });
    });
    \end{lstlisting}

    \subsection{Casos de Prueba Unitaria - Frontend (Jest + RTL)}

    \subsubsection{TC-FE-U01: Componente BiometricLogin Renderiza Correctamente}

    \begin{table}[H]
    \centering
    \caption{TC-FE-U01: Renderizado de BiometricLogin}
    \label{tab:tc_fe_u01}
    \begin{tabularx}{\textwidth}{|l|X|}
    \hline
    \textbf{ID} & TC-FE-U01 \\ \hline
    \textbf{Nombre} & Verificar renderizado correcto de BiometricLogin component \\ \hline
    \textbf{Componente} & BiometricLoginAdvanced.jsx \\ \hline
    \textbf{Prioridad} & Media \\ \hline
    \textbf{Precondiciones} & 
    \begin{itemize}[leftmargin=*, nosep, after=\vspace{-\baselineskip}]
        \item React Testing Library configurado
        \item face-api.js mockeado
    \end{itemize} \\ \hline
    \textbf{Pasos} & 
    \begin{enumerate}[leftmargin=*, nosep, after=\vspace{-\baselineskip}]
        \item Renderizar componente BiometricLoginAdvanced
        \item Buscar elementos clave del DOM
        \item Verificar visibilidad de botones
    \end{enumerate} \\ \hline
    \textbf{Resultado Esperado} & 
    \begin{itemize}[leftmargin=*, nosep, after=\vspace{-\baselineskip}]
        \item Input de email visible
        \item Botón "Start Face Detection" visible
        \item Canvas de video presente en DOM
    \end{itemize} \\ \hline
    \textbf{Resultado Obtenido} & \textit{Pendiente de ejecución} \\ \hline
    \textbf{Estado} & No Ejecutado \\ \hline
    \end{tabularx}
    \end{table}

    \textbf{Script de Prueba:}

    \begin{lstlisting}[language=JavaScript, caption={TC-FE-U01: RTL Script}]
    import { render, screen } from "@testing-library/react";
    import { BrowserRouter } from "react-router-dom";
    import BiometricLoginAdvanced from "../src/components/BiometricLoginAdvanced";

    // Mock face-api.js
    jest.mock("face-api.js", () => ({
    nets: {
        tinyFaceDetector: { loadFromUri: jest.fn() },
        faceLandmark68Net: { loadFromUri: jest.fn() },
        faceExpressionNet: { loadFromUri: jest.fn() }
    },
    detectSingleFace: jest.fn(),
    TinyFaceDetectorOptions: jest.fn()
    }));

    describe("TC-FE-U01: BiometricLogin Component Rendering", () => {
    it("should render email input field", () => {
        render(
        <BrowserRouter>
            <BiometricLoginAdvanced />
        </BrowserRouter>
        );

        const emailInput = screen.getByLabelText(/email/i);
        expect(emailInput).toBeInTheDocument();
        expect(emailInput).toHaveAttribute("type", "email");
    });

    it("should render start detection button", () => {
        render(
        <BrowserRouter>
            <BiometricLoginAdvanced />
        </BrowserRouter>
        );

        const startButton = screen.getByText(/start face detection/i);
        expect(startButton).toBeInTheDocument();
        expect(startButton).toBeEnabled();
    });

    it("should render video canvas element", () => {
        const { container } = render(
        <BrowserRouter>
            <BiometricLoginAdvanced />
        </BrowserRouter>
        );

        const canvas = container.querySelector("canvas");
        expect(canvas).toBeInTheDocument();
    });
    });
    \end{lstlisting}

    % ============================================================
    % 5. DISEÑO DE CASOS DE PRUEBA DE INTEGRACIÓN
    % ============================================================
    \section{Diseño de Casos de Prueba de Integración}

    \subsection{Casos de Prueba de Integración - APIs (Postman)}

    \subsubsection{TC-AUTH-003: Validación de Email Inválido}

    \begin{table}[H]
    \centering
    \caption{TC-AUTH-003: Email Inválido}
    \label{tab:tc_auth_003}
    \begin{tabularx}{\textwidth}{|l|X|}
    \hline
    \textbf{ID} & TC-AUTH-003 \\ \hline
    \textbf{Nombre} & Registro rechaza email con formato inválido \\ \hline
    \textbf{Endpoint} & POST /api/auth/register \\ \hline
    \textbf{Prioridad} & Alta \\ \hline
    \textbf{Precondiciones} & 
    \begin{itemize}[leftmargin=*, nosep, after=\vspace{-\baselineskip}]
        \item Backend desplegado en http://localhost:4000
        \item Base de datos accesible
    \end{itemize} \\ \hline
    \textbf{Headers} & 
    \begin{minipage}[t]{\linewidth}
\begin{lstlisting}
Content-Type: application/json
\end{lstlisting}
    \end{minipage} \\ \hline
    \textbf{Body (JSON)} & 
    \begin{minipage}[t]{\linewidth}
\begin{lstlisting}
{
  "email": "invalid-email-format",
  "password": "Test123!",
  "username": "testuser"
}
\end{lstlisting}
    \end{minipage} \\ \hline
    \textbf{Resultado Esperado} & 
    \begin{itemize}[leftmargin=*, nosep, after=\vspace{-\baselineskip}]
        \item Status Code: 400 Bad Request
        \item Response: \{"success": false, "message": "Invalid email format"\}
        \item No se crea usuario en BD
    \end{itemize} \\ \hline
    \textbf{Test Script (Postman)} & 
    \begin{minipage}[t]{\linewidth}
\begin{lstlisting}[language=JavaScript]
pm.test("Status is 400", () => {
  pm.response.to.have.status(400);
});

pm.test("Error message present", () => {
  const json = pm.response.json();
  pm.expect(json.success).to.be.false;
  pm.expect(json.message).to.include("email");
});
\end{lstlisting}
    \end{minipage} \\ \hline
    \textbf{Resultado Obtenido} & \textit{Pendiente de ejecución} \\ \hline
    \textbf{Estado} & No Ejecutado \\ \hline
    \end{tabularx}
    \end{table}

    \subsubsection{TC-BIO-003: Extracción de Encoding 128D}

    \begin{table}[H]
    \centering
    \caption{TC-BIO-003: Extracción de Encoding}
    \label{tab:tc_bio_003}
    \begin{tabularx}{\textwidth}{|l|X|}
    \hline
    \textbf{ID} & TC-BIO-003 \\ \hline
    \textbf{Nombre} & Microservicio extrae encoding 128D correctamente de imagen facial \\ \hline
    \textbf{Endpoint} & POST \url{http://localhost:8001/extract-features} (Interno) \\ \hline
    \textbf{Prioridad} & Alta \\ \hline
    \textbf{Precondiciones} & 
    \begin{itemize}[leftmargin=*, nosep, after=\vspace{-\baselineskip}]
        \item Facial service desplegado
        \item Backend tiene acceso a red interna
        \item INTERNAL\_SERVICE\_TOKEN configurado
    \end{itemize} \\ \hline
    \textbf{Headers} & 
    \begin{minipage}[t]{\linewidth}
\begin{lstlisting}[escapeinside={(*@}{@*)}]
X-Internal-Token: (*@\{\{@*)internal_service_token(*@\}\}@*)
\end{lstlisting}
    \end{minipage} \\ \hline
    \textbf{Body (multipart)} & 
    \begin{itemize}[leftmargin=*, nosep, after=\vspace{-\baselineskip}]
        \item face: [archivo face.jpg válido]
    \end{itemize} \\ \hline
    \textbf{Resultado Esperado} & 
    \begin{itemize}[leftmargin=*, nosep, after=\vspace{-\baselineskip}]
        \item Status Code: 200 OK
        \item Response contiene encoding: array de 128 floats
        \item liveness score: 0.0-1.0
        \item quality score: 0.0-1.0
        \item confidence: 0.0-1.0
    \end{itemize} \\ \hline
    \textbf{Test Script} & 
    \begin{minipage}[t]{\linewidth}
\begin{lstlisting}[language=JavaScript]
pm.test("Status is 200", () => {
  pm.response.to.have.status(200);
});

pm.test("Encoding is 128D array", () => {
  const json = pm.response.json();
  pm.expect(json.encoding).to.be.an("array");
  pm.expect(json.encoding).to.have.lengthOf(128);
});

pm.test("Liveness score present", () => {
  const json = pm.response.json();
  pm.expect(json.liveness_score).to.be.a("number");
  pm.expect(json.liveness_score).to.be.within(0, 1);
});
\end{lstlisting}
    \end{minipage} \\ \hline
    \textbf{Resultado Obtenido} & \textit{Pendiente de ejecución} \\ \hline
    \textbf{Estado} & No Ejecutado \\ \hline
    \end{tabularx}
    \end{table}

    \subsubsection{TC-TRIP-006: Protección IDOR en Viajes}

    \begin{table}[H]
    \centering
    \caption{TC-TRIP-006: Protección IDOR}
    \label{tab:tc_trip_006}
    \begin{tabularx}{\textwidth}{|l|X|}
    \hline
    \textbf{ID} & TC-TRIP-006 \\ \hline
    \textbf{Nombre} & Usuario no puede acceder a viajes de otros usuarios (IDOR prevention) \\ \hline
    \textbf{Endpoint} & GET /trips/:tripId \\ \hline
    \textbf{Prioridad} & Alta \\ \hline
    \textbf{Precondiciones} & 
    \begin{itemize}[leftmargin=*, nosep, after=\vspace{-\baselineskip}]
        \item Usuario A autenticado (JWT token en variable)
        \item Usuario B tiene viaje con \_id conocido
    \end{itemize} \\ \hline
    \textbf{Headers} & 
    \begin{minipage}[t]{\linewidth}
\begin{lstlisting}[escapeinside={(*@}{@*)}]
Authorization: Bearer (*@\{\{@*)jwt_token_userA(*@\}\}@*)
\end{lstlisting}
    \end{minipage} \\ \hline
    \textbf{URL} & 
    \begin{minipage}[t]{\linewidth}
\begin{lstlisting}[escapeinside={(*@}{@*)}]
GET http://localhost:4000/trips/(*@\{\{@*)trip_id_userB(*@\}\}@*)
\end{lstlisting}
    \end{minipage} \\ \hline
    \textbf{Resultado Esperado} & 
    \begin{itemize}[leftmargin=*, nosep, after=\vspace{-\baselineskip}]
        \item Status Code: 403 Forbidden
        \item Response: \{"success": false, "message": "Access denied"\}
        \item No se retornan datos del viaje ajeno
    \end{itemize} \\ \hline
    \textbf{Test Script} & 
    \begin{minipage}[t]{\linewidth}
\begin{lstlisting}[language=JavaScript]
pm.test("Status is 403", () => {
  pm.response.to.have.status(403);
});

pm.test("Access denied message", () => {
  const json = pm.response.json();
  pm.expect(json.success).to.be.false;
  pm.expect(json.message).to.include("denied");
});
\end{lstlisting}
    \end{minipage} \\ \hline
    \textbf{Resultado Obtenido} & \textit{Pendiente de ejecución} \\ \hline
    \textbf{Estado} & No Ejecutado \\ \hline
    \end{tabularx}
    \end{table}

    % ============================================================
    % 6. DISEÑO DE CASOS DE PRUEBA E2E
    % ============================================================
    \section{Diseño de Casos de Prueba End-to-End}

    \subsection{Casos de Prueba E2E - Frontend (Cypress)}

    \subsubsection{TC-AUTH-005: Login Tradicional Exitoso}

    \begin{table}[H]
    \centering
    \caption{TC-AUTH-005: Login Exitoso}
    \label{tab:tc_auth_005}
    \begin{tabularx}{\textwidth}{|l|X|}
    \hline
    \textbf{ID} & TC-AUTH-005 \\ \hline
    \textbf{Nombre} & Login tradicional exitoso con credenciales válidas \\ \hline
    \textbf{Flujo} & Autenticación \\ \hline
    \textbf{Prioridad} & Alta \\ \hline
    \textbf{Precondiciones} & 
    \begin{itemize}[leftmargin=*, nosep, after=\vspace{-\baselineskip}]
        \item Usuario registrado: test@mail.com / Test123!
        \item Frontend accesible en http://localhost:3001
        \item Backend funcional
    \end{itemize} \\ \hline
    \textbf{Pasos} & 
    \begin{enumerate}[leftmargin=*, nosep, after=\vspace{-\baselineskip}]
        \item Navegar a http://localhost:3001/login
        \item Ingresar email: test@mail.com
        \item Ingresar password: Test123!
        \item Hacer clic en botón "Login"
        \item Esperar redirección
    \end{enumerate} \\ \hline
    \textbf{Resultado Esperado} & 
    \begin{itemize}[leftmargin=*, nosep, after=\vspace{-\baselineskip}]
        \item Redirección a /dashboard
        \item Mensaje de bienvenida visible
        \item Token JWT almacenado en localStorage
        \item Navbar muestra nombre del usuario
    \end{itemize} \\ \hline
    \textbf{Resultado Obtenido} & \textit{Pendiente de ejecución} \\ \hline
    \textbf{Estado} & No Ejecutado \\ \hline
    \end{tabularx}
    \end{table}

    \textbf{Script Cypress:}

    \begin{lstlisting}[language=JavaScript, caption={TC-AUTH-005: Cypress Script}]
    describe("TC-AUTH-005: Traditional Login Success", () => {
    beforeEach(() => {
        // Clear storage
        cy.clearLocalStorage();
        cy.clearCookies();
        
        // Visit login page
        cy.visit("http://localhost:3001/login");
    });

    it("should login successfully with valid credentials", () => {
        // Step 1: Enter email
        cy.get("input[name="email"]").type("test@mail.com");
        
        // Step 2: Enter password
        cy.get("input[name="password"]").type("Test123!");
        
        // Step 3: Click login button
        cy.get("button[type="submit"]").click();
        
        // Assertions
        // Step 4: Verify redirect to dashboard
        cy.url().should("include", "/dashboard");
        
        // Step 5: Verify welcome message
        cy.contains("Welcome").should("be.visible");
        
        // Step 6: Verify JWT token in localStorage
        cy.window().its("localStorage.token").should("exist");
        
        // Step 7: Verify user name in navbar
        cy.get(".navbar").contains("test").should("be.visible");
    });

    it("should include Authorization header in subsequent requests", () => {
        // Login
        cy.get("input[name="email"]").type("test@mail.com");
        cy.get("input[name="password"]").type("Test123!");
        cy.get("button[type="submit"]").click();
        
        // Intercept API call
        cy.intercept("GET", "/api/trips").as("getTrips");
        
        // Navigate to trips page
        cy.visit("http://localhost:3001/trips");
        
        // Wait for request
        cy.wait("@getTrips").its("request.headers").should("have.property", "authorization");
    });
    });
    \end{lstlisting}

    \subsubsection{TC-BIO-006: Login Biométrico Exitoso}

    \begin{table}[H]
    \centering
    \caption{TC-BIO-006: Login Biométrico}
    \label{tab:tc_bio_006}
    \begin{tabularx}{\textwidth}{|l|X|}
    \hline
    \textbf{ID} & TC-BIO-006 \\ \hline
    \textbf{Nombre} & Login biométrico exitoso con rostro registrado \\ \hline
    \textbf{Flujo} & Autenticación Biométrica \\ \hline
    \textbf{Prioridad} & Alta \\ \hline
    \textbf{Precondiciones} & 
    \begin{itemize}[leftmargin=*, nosep, after=\vspace{-\baselineskip}]
        \item Usuario con biometría registrada: bio@mail.com
        \item Cámara disponible o mock de video
        \item face-api.js modelos cargados
    \end{itemize} \\ \hline
    \textbf{Pasos} & 
    \begin{enumerate}[leftmargin=*, nosep, after=\vspace{-\baselineskip}]
        \item Navegar a /login
        \item Hacer clic en "Login with Face"
        \item Ingresar email: bio@mail.com
        \item Hacer clic en "Start Face Detection"
        \item Permitir acceso a cámara
        \item Esperar detección de rostro (verde)
        \item Realizar 2 parpadeos detectados
        \item Esperar captura (3..2..1..)
        \item Verificar resultado
    \end{enumerate} \\ \hline
    \textbf{Resultado Esperado} & 
    \begin{itemize}[leftmargin=*, nosep, after=\vspace{-\baselineskip}]
        \item Barra de liveness muestra 100\%
        \item Mensaje "Authentication successful!"
        \item Redirección a /dashboard
        \item Token JWT en localStorage
    \end{itemize} \\ \hline
    \textbf{Resultado Obtenido} & \textit{Pendiente de ejecución} \\ \hline
    \textbf{Estado} & No Ejecutado \\ \hline
    \end{tabularx}
    \end{table}

    \textbf{Nota:} Este caso requiere configuración especial de Cypress para mocking de MediaDevices.getUserMedia().

    \subsubsection{TC-TRIP-001: Crear Viaje Exitosamente}

    \begin{table}[H]
    \centering
    \caption{TC-TRIP-001: Crear Viaje}
    \label{tab:tc_trip_001}
    \begin{tabularx}{\textwidth}{|l|X|}
    \hline
    \textbf{ID} & TC-TRIP-001 \\ \hline
    \textbf{Nombre} & Crear viaje exitosamente con datos completos \\ \hline
    \textbf{Flujo} & Gestión de Viajes \\ \hline
    \textbf{Prioridad} & Alta \\ \hline
    \textbf{Precondiciones} & 
    \begin{itemize}[leftmargin=*, nosep, after=\vspace{-\baselineskip}]
        \item Usuario autenticado
        \item En página /trips
    \end{itemize} \\ \hline
    \textbf{Datos de Prueba} & 
    \begin{itemize}[leftmargin=*, nosep, after=\vspace{-\baselineskip}]
        \item Título: "Vacaciones Verano 2026"
        \item Destino: "Madrid, España"
        \item Fecha inicio: 2026-07-01
        \item Fecha fin: 2026-07-15
        \item Presupuesto: 3000
        \item Descripción: "Viaje familiar a Europa"
    \end{itemize} \\ \hline
    \textbf{Pasos} & 
    \begin{enumerate}[leftmargin=*, nosep, after=\vspace{-\baselineskip}]
        \item Hacer clic en "New Trip"
        \item Completar formulario con datos de prueba
        \item Hacer clic en "Save"
        \item Esperar respuesta
    \end{enumerate} \\ \hline
    \textbf{Resultado Esperado} & 
    \begin{itemize}[leftmargin=*, nosep, after=\vspace{-\baselineskip}]
        \item Mensaje de éxito visible
        \item Viaje aparece en lista
        \item Datos coinciden con ingresados
        \item Nuevo viaje tiene \_id asignado
    \end{itemize} \\ \hline
    \textbf{Resultado Obtenido} & \textit{Pendiente de ejecución} \\ \hline
    \textbf{Estado} & No Ejecutado \\ \hline
    \end{tabularx}
    \end{table}

    \textbf{Script Cypress:}

    \begin{lstlisting}[language=JavaScript, caption={TC-TRIP-001: Cypress Script}]
    describe("TC-TRIP-001: Create Trip Successfully", () => {
    beforeEach(() => {
        // Login first
        cy.login("test@mail.com", "Test123!"); // Custom command
        cy.visit("http://localhost:3001/trips");
    });

    it("should create new trip with complete data", () => {
        // Step 1: Click new trip button
        cy.contains("New Trip").click();
        
        // Step 2: Fill form
        cy.get("input[name="title"]").type("Vacaciones Verano 2026");
        cy.get("input[name="destination"]").type("Madrid, España");
        cy.get("input[name="startDate"]").type("2026-07-01");
        cy.get("input[name="endDate"]").type("2026-07-15");
        cy.get("input[name="budget"]").type("3000");
        cy.get("textarea[name="description"]").type("Viaje familiar a Europa");
        
        // Step 3: Submit
        cy.get("button[type="submit"]").click();
        
        // Assertions
        // Step 4: Verify success message
        cy.contains("Trip created successfully").should("be.visible");
        
        // Step 5: Verify trip in list
        cy.contains("Vacaciones Verano 2026").should("be.visible");
        cy.contains("Madrid, España").should("be.visible");
        
        // Step 6: Verify data persistence
        cy.reload();
        cy.contains("Vacaciones Verano 2026").should("be.visible");
    });

    it("should validate startDate < endDate", () => {
        cy.contains("New Trip").click();
        
        cy.get("input[name="startDate"]").type("2026-07-15");
        cy.get("input[name="endDate"]").type("2026-07-01");
        cy.get("button[type="submit"]").click();
        
        // Should show error
        cy.contains("End date must be after start date").should("be.visible");
    });
    });
    \end{lstlisting}

    % ============================================================
    % 7. DISEÑO DE CASOS DE PRUEBA DE SEGURIDAD
    % ============================================================
    \section{Diseño de Casos de Prueba de Seguridad}

    \subsection{Casos de Prueba de Seguridad - OWASP ZAP}

    \subsubsection{TC-SEC-001: Protección IDOR}

    \begin{table}[H]
    \centering
    \caption{TC-SEC-001: IDOR Prevention}
    \label{tab:tc_sec_001}
    \begin{tabularx}{\textwidth}{|l|X|}
    \hline
    \textbf{ID} & TC-SEC-001 \\ \hline
    \textbf{Nombre} & Prevención de IDOR (Insecure Direct Object Reference) en /trips/:id \\ \hline
    \textbf{Vulnerabilidad} & OWASP A01:2021 - Broken Access Control \\ \hline
    \textbf{Prioridad} & Alta \\ \hline
    \textbf{Método} & Escaneo Activo + Manual \\ \hline
    \textbf{Precondiciones} & 
    \begin{itemize}[leftmargin=*, nosep, after=\vspace{-\baselineskip}]
        \item OWASP ZAP configurado como proxy
        \item 2 usuarios con sesiones activas (UserA, UserB)
        \item UserB tiene viaje con \_id conocido
    \end{itemize} \\ \hline
    \textbf{Pasos} & 
    \begin{enumerate}[leftmargin=*, nosep, after=\vspace{-\baselineskip}]
        \item Autenticar como UserA en ZAP
        \item Interceptar GET /trips/:id con token de UserA
        \item Modificar :id por ID de viaje de UserB
        \item Enviar request
        \item Analizar respuesta
    \end{enumerate} \\ \hline
    \textbf{Resultado Esperado} & 
    \begin{itemize}[leftmargin=*, nosep, after=\vspace{-\baselineskip}]
        \item HTTP 403 Forbidden
        \item No se retornan datos del viaje ajeno
        \item No hay information leakage
    \end{itemize} \\ \hline
    \textbf{Criterio de Éxito} & Sin vulnerabilidad IDOR detectada \\ \hline
    \textbf{Resultado Obtenido} & \textit{Pendiente de ejecución} \\ \hline
    \textbf{Estado} & No Ejecutado \\ \hline
    \end{tabularx}
    \end{table}

    \subsubsection{TC-SEC-003: Cross-Site Scripting (XSS)}

    \begin{table}[H]
    \centering
    \caption{TC-SEC-003: XSS Detection}
    \label{tab:tc_sec_003}
    \begin{tabularx}{\textwidth}{|l|X|}
    \hline
    \textbf{ID} & TC-SEC-003 \\ \hline
    \textbf{Nombre} & Detección de vulnerabilidades XSS en formularios de entrada \\ \hline
    \textbf{Vulnerabilidad} & OWASP A03:2021 - Injection \\ \hline
    \textbf{Prioridad} & Alta \\ \hline
    \textbf{Método} & Escaneo Activo (ZAP Spider + Active Scan) \\ \hline
    \textbf{Targets} & 
    \begin{itemize}[leftmargin=*, nosep, after=\vspace{-\baselineskip}]
        \item /register (username, email)
        \item /trips (title, description, destination)
        \item /weather (city search)
    \end{itemize} \\ \hline
    \textbf{Payloads} & 
    \begin{minipage}[t]{\linewidth}
\begin{lstlisting}
<script>alert("XSS")</script>
<img src=x onerror=alert("XSS")>
"><script>alert(String.fromCharCode(88,83,83))</script>
\end{lstlisting}
    \end{minipage} \\ \hline
    \textbf{Pasos} & 
    \begin{enumerate}[leftmargin=*, nosep, after=\vspace{-\baselineskip}]
        \item Configurar contexto en ZAP
        \item Ejecutar Spider en http://localhost:3001
        \item Ejecutar Active Scan con XSS rules
        \item Revisar alertas generadas
        \item Verificar manualmente payloads sospechosos
    \end{enumerate} \\ \hline
    \textbf{Resultado Esperado} & 
    \begin{itemize}[leftmargin=*, nosep, after=\vspace{-\baselineskip}]
        \item 0 alertas de XSS de severidad Alta/Crítica
        \item Inputs sanitizados correctamente
        \item React escapa HTML automáticamente
    \end{itemize} \\ \hline
    \textbf{Criterio de Éxito} & Sin XSS ejecutable detectado \\ \hline
    \textbf{Resultado Obtenido} & \textit{Pendiente de ejecución} \\ \hline
    \textbf{Estado} & No Ejecutado \\ \hline
    \end{tabularx}
    \end{table}

    \subsubsection{TC-SEC-005: Headers de Seguridad}

    \begin{table}[H]
    \centering
    \caption{TC-SEC-005: Security Headers}
    \label{tab:tc_sec_005}
    \begin{tabularx}{\textwidth}{|l|X|}
    \hline
    \textbf{ID} & TC-SEC-005 \\ \hline
    \textbf{Nombre} & Verificación de headers de seguridad HTTP \\ \hline
    \textbf{Categoría} & Security Misconfiguration \\ \hline
    \textbf{Prioridad} & Media \\ \hline
    \textbf{Método} & Passive Scan (ZAP) \\ \hline
    \textbf{Targets} & Todas las respuestas HTTP del backend \\ \hline
    \textbf{Headers Requeridos} & 
    \begin{itemize}[leftmargin=*, nosep, after=\vspace{-\baselineskip}]
        \item Strict-Transport-Security
        \item X-Content-Type-Options: nosniff
        \item X-Frame-Options: DENY
        \item Content-Security-Policy
        \item X-XSS-Protection: 1; mode=block
    \end{itemize} \\ \hline
    \textbf{Pasos} & 
    \begin{enumerate}[leftmargin=*, nosep, after=\vspace{-\baselineskip}]
        \item Navegar aplicación con ZAP proxy activo
        \item ZAP registra headers automáticamente
        \item Revisar alertas de "Missing Security Headers"
        \item Verificar cada header manualmente
    \end{enumerate} \\ \hline
    \textbf{Resultado Esperado} & 
    \begin{itemize}[leftmargin=*, nosep, after=\vspace{-\baselineskip}]
        \item Todos los headers requeridos presentes
        \item HSTS con maxAge $\geq$ 31536000
        \item CSP configurado correctamente
    \end{itemize} \\ \hline
    \textbf{Criterio de Éxito} & Sin alertas de headers faltantes \\ \hline
    \textbf{Resultado Obtenido} & \textit{Pendiente de ejecución} \\ \hline
    \textbf{Estado} & No Ejecutado \\ \hline
    \end{tabularx}
    \end{table}

    % ============================================================
    % 8. MATRIZ RESUMEN DE CASOS DE PRUEBA
    % ============================================================
    \section{Matriz Resumen de Casos de Prueba}

    \begin{table}[H]
    \centering
    \caption{Resumen de Casos de Prueba por Nivel}
    \label{tab:resumen_casos}
    \begin{tabularx}{\textwidth}{|l|c|c|c|c|}
    \hline
    \textbf{Nivel de Prueba} & \textbf{Total Casos} & \textbf{Alta Prioridad} & \textbf{Media} & \textbf{Baja} \\ \hline
    Unitarias (Backend) & 18 & 12 & 5 & 1 \\ \hline
    Unitarias (Frontend) & 15 & 8 & 7 & 0 \\ \hline
    Integración (APIs) & 28 & 18 & 8 & 2 \\ \hline
    E2E (Cypress) & 16 & 10 & 5 & 1 \\ \hline
    Seguridad (ZAP) & 10 & 7 & 3 & 0 \\ \hline
    \textbf{TOTAL} & \textbf{87} & \textbf{55} & \textbf{28} & \textbf{4} \\ \hline
    \end{tabularx}
    \end{table}

    \begin{table}[H]
    \centering
    \caption{Distribución por Módulo}
    \label{tab:casos_modulo}
    \begin{tabularx}{\textwidth}{|X|c|c|}
    \hline
    \textbf{Módulo} & \textbf{Casos Diseñados} & \textbf{Cobertura Estimada} \\ \hline
    Autenticación Tradicional & 12 & 85\% \\ \hline
    Autenticación Biométrica & 18 & 80\% \\ \hline
    Gestión de Viajes (CRUD) & 16 & 90\% \\ \hline
    Clima & 8 & 70\% \\ \hline
    Administración & 6 & 75\% \\ \hline
    Seguridad General & 10 & Variable \\ \hline
    Microservicio Facial & 9 & 80\% \\ \hline
    Middlewares & 8 & 85\% \\ \hline
    \textbf{TOTAL} & \textbf{87} & \textbf{82\% promedio} \\ \hline
    \end{tabularx}
    \end{table}

    % ============================================================
    % 9. PLANTILLA DE REPORTE DE DEFECTOS
    % ============================================================
    \section{Plantilla de Reporte de Defectos}

    \begin{table}[H]
    \centering
    \caption{Plantilla de Reporte de Defectos}
    \label{tab:plantilla_defecto}
    \begin{tabularx}{\textwidth}{|l|X|}
    \hline
    \textbf{Campo} & \textbf{Descripción} \\ \hline
    \textbf{ID Defecto} & Identificador único (formato: BUG-XXX) \\ \hline
    \textbf{Título} & Resumen corto del problema \\ \hline
    \textbf{Módulo} & Componente afectado \\ \hline
    \textbf{Severidad} & Crítica / Alta / Media / Baja \\ \hline
    \textbf{Prioridad} & P1 (Urgente) / P2 (Alta) / P3 (Media) / P4 (Baja) \\ \hline
    \textbf{Caso de Prueba} & ID del caso que detectó el defecto \\ \hline
    \textbf{Precondiciones} & Estado del sistema antes del error \\ \hline
    \textbf{Pasos para Reproducir} & Secuencia exacta para replicar el bug \\ \hline
    \textbf{Resultado Esperado} & Comportamiento correcto esperado \\ \hline
    \textbf{Resultado Actual} & Comportamiento erróneo observado \\ \hline
    \textbf{Evidencia} & Screenshots, logs, videos \\ \hline
    \textbf{Entorno} & OS, Browser, Node version, etc. \\ \hline
    \textbf{Reportado por} & Nombre del tester \\ \hline
    \textbf{Fecha Detección} & Timestamp de detección \\ \hline
    \textbf{Asignado a} & Desarrollador responsable \\ \hline
    \textbf{Estado} & Abierto / En Progreso / Resuelto / Cerrado / Reabierto \\ \hline
    \textbf{Fecha Resolución} & Timestamp de resolución \\ \hline
    \textbf{Comentarios} & Notas adicionales, workarounds \\ \hline
    \end{tabularx}
    \end{table}

    \subsection{Ejemplo de Defecto}

    \begin{table}[H]
    \centering
    \caption{Ejemplo de Reporte de Defecto}
    \label{tab:ejemplo_defecto}
    \begin{tabularx}{\textwidth}{|l|X|}
    \hline
    \textbf{ID} & BUG-001 \\ \hline
    \textbf{Título} & Error 500 al crear viaje con fecha de inicio en el pasado \\ \hline
    \textbf{Módulo} & Backend - tripController.js \\ \hline
    \textbf{Severidad} & Media \\ \hline
    \textbf{Prioridad} & P2 (Alta) \\ \hline
    \textbf{Caso de Prueba} & TC-TRIP-002 \\ \hline
    \textbf{Precondiciones} & Usuario autenticado, en formulario de crear viaje \\ \hline
    \textbf{Pasos para Reproducir} & 
    \begin{enumerate}[leftmargin=*, nosep, after=\vspace{-\baselineskip}]
        \item Login como test@mail.com
        \item Navegar a /trips
        \item Clic en "New Trip"
        \item Ingresar startDate: 2025-01-01 (fecha pasada)
        \item Ingresar endDate: 2025-01-10
        \item Clic en "Save"
    \end{enumerate} \\ \hline
    \textbf{Resultado Esperado} & Mensaje de validación: "Start date cannot be in the past" \\ \hline
    \textbf{Resultado Actual} & Error 500 Internal Server Error, no mensaje al usuario \\ \hline
    \textbf{Evidencia} & Screenshot adjunto, log: TypeError at tripController:45 \\ \hline
    \textbf{Entorno} & Windows 11, Chrome 120, Node 18.17.0 \\ \hline
    \textbf{Reportado por} & Anthony Villareal \\ \hline
    \textbf{Fecha Detección} & 2026-01-15 10:30 \\ \hline
    \textbf{Asignado a} & Cáceres Germán \\ \hline
    \textbf{Estado} & \textit{Pendiente de ejecución - Ejemplo ilustrativo} \\ \hline
    \end{tabularx}
    \end{table}

    % ============================================================
    % 10. MÉTRICAS DE DISEÑO
    % ============================================================
    \section{Métricas de Diseño}

    \begin{table}[H]
    \centering
    \caption{Métricas de Cobertura de Requisitos}
    \label{tab:metricas_cobertura}
    \begin{tabularx}{\textwidth}{|X|c|c|c|}
    \hline
    \textbf{Categoría} & \textbf{Total Req.} & \textbf{Casos Diseñados} & \textbf{\% Cobertura} \\ \hline
    Requisitos Funcionales & 29 & 63 & 100\% \\ \hline
    Requisitos de Seguridad & 8 & 10 & 100\% \\ \hline
    Requisitos No Funcionales & 5 & 4 & 80\% \\ \hline
    \textbf{TOTAL} & \textbf{42} & \textbf{77} & \textbf{98\%} \\ \hline
    \end{tabularx}
    \end{table}

    \textbf{Nota:} Algunos requisitos tienen múltiples casos de prueba asociados (enfoque de pruebas exhaustivas).

    \begin{table}[H]
    \centering
    \caption{Estimación de Esfuerzo de Ejecución}
    \label{tab:esfuerzo_ejecucion}
    \begin{tabularx}{\textwidth}{|X|c|c|c|}
    \hline
    \textbf{Nivel de Prueba} & \textbf{Casos} & \textbf{Tiempo Prom.} & \textbf{Total Horas} \\ \hline
    Unitarias (Automatizadas) & 33 & 2 min & 1.1h \\ \hline
    Integración (Automatizadas) & 28 & 1 min & 0.5h \\ \hline
    E2E (Automatizadas) & 16 & 3 min & 0.8h \\ \hline
    Seguridad (Automatizada) & 10 & Variable & 6h \\ \hline
    Análisis Manual & - & - & 4h \\ \hline
    \textbf{TOTAL} & \textbf{87} & - & \textbf{12.4h} \\ \hline
    \end{tabularx}
    \end{table}

    % ============================================================
    % 11. CONCLUSIONES
    % ============================================================
    \section{Conclusiones}

    \subsection{Resumen del Diseño}

    Se han diseñado \textbf{87 casos de prueba} que cubren:

    \begin{itemize}
        \item \textbf{4 niveles de prueba:} Unitarias, Integración, E2E y Seguridad
        \item \textbf{42 requisitos funcionales y no funcionales} (98\% de cobertura)
        \item \textbf{7 módulos críticos} del sistema TravelBrain
        \item \textbf{Trazabilidad completa} requisitos $\leftrightarrow$ casos de prueba
    \end{itemize}

    \subsection{Fortalezas del Diseño}

    \begin{enumerate}
        \item \textbf{Cobertura Exhaustiva:} 98\% de requisitos cubiertos con casos de prueba
        \item \textbf{Automatización:} 88\% de casos automatizables (unitarias, integración, E2E)
        \item \textbf{Trazabilidad:} Matriz completa que vincula requisitos con casos
        \item \textbf{Priorización:} 63\% casos de alta prioridad enfocados en funcionalidades críticas
        \item \textbf{Seguridad:} Cobertura específica de OWASP Top 10
        \item \textbf{Reutilización:} Scripts documentados pueden ejecutarse en regresión
    \end{enumerate}

    \subsection{Próximos Pasos}

    \begin{enumerate}
        \item \textbf{Semana 2:} Implementación de scripts automatizados
        \item \textbf{Semana 2-3:} Ejecución de todos los casos diseñados
        \item \textbf{Semana 3:} Generación de Test Summary Report con resultados reales
        \item \textbf{Entrega Final:} Consolidación de evidencias y métricas
    \end{enumerate}

    \subsection{Expectativas}

    Se espera que la ejecución de estos 87 casos de prueba permita:

    \begin{itemize}
        \item Detectar defectos críticos antes de producción
        \item Alcanzar cobertura de código $\geq$ 70\% en módulos críticos
        \item Identificar vulnerabilidades de seguridad de nivel ALTO/CRÍTICO
        \item Validar funcionalidad completa de autenticación biométrica
        \item Generar base de pruebas de regresión para mantenimiento futuro
    \end{itemize}

    % ============================================================
    % REFERENCIAS
    % ============================================================
    \newpage
    \printbibliography[title={Referencias Bibliográficas}]

    \begin{filecontents}{referencias.bib}
    @techreport{ieee829,
    author = {{IEEE}},
    title = {IEEE Standard for Software and System Test Documentation},
    institution = {Institute of Electrical and Electronics Engineers},
    year = {2008},
    number = {IEEE Std 829-2008}
    }

    @techreport{iso29119,
    author = {{ISO/IEC/IEEE}},
    title = {Software and systems engineering -- Software testing -- Part 3: Test documentation},
    institution = {International Organization for Standardization},
    year = {2013},
    number = {ISO/IEC/IEEE 29119-3:2013}
    }

    @book{jorgensen2013,
    author = {Jorgensen, Paul C.},
    title = {Software Testing: A Craftsman"s Approach},
    edition = {4th},
    publisher = {CRC Press},
    year = {2013},
    isbn = {978-1466560680}
    }

    @misc{owasp2021,
    author = {{OWASP Foundation}},
    title = {OWASP Top 10 - 2021},
    year = {2021},
    url = {https://owasp.org/Top10/},
    note = {Accessed: 2026-01-21}
    }
    \end{filecontents}

    % ============================================================
    % ANEXOS
    % ============================================================
    \newpage
    \appendix
    \section{Anexos}

    \subsection{Anexo A: Configuración de Cypress}

    \begin{lstlisting}[language=JavaScript, caption={cypress.config.js}]
    const { defineConfig } = require("cypress");

    module.exports = defineConfig({
    e2e: {
        baseUrl: "http://localhost:3001",
        specPattern: "cypress/e2e/**/*.cy.{js,jsx,ts,tsx}",
        supportFile: "cypress/support/e2e.js",
        video: true,
        screenshotOnRunFailure: true,
        viewportWidth: 1280,
        viewportHeight: 720,
        defaultCommandTimeout: 10000,
        requestTimeout: 10000,
        responseTimeout: 10000,
        env: {
        apiUrl: "http://localhost:4000"
        }
    }
    });
    \end{lstlisting}

    \subsection{Anexo B: Comandos Personalizados Cypress}

    \begin{lstlisting}[language=JavaScript, caption={cypress/support/commands.js}]
    // Custom login command
    Cypress.Commands.add("login", (email, password) => {
    cy.request({
        method: "POST",
        url: `${Cypress.env("apiUrl")}/api/auth/login`,
        body: { email, password }
    }).then((response) => {
        window.localStorage.setItem("token", response.body.data.token);
    });
    });

    // Custom logout command
    Cypress.Commands.add("logout", () => {
    window.localStorage.removeItem("token");
    });
    \end{lstlisting}

    \subsection{Anexo C: Configuración de Postman Collection}

    \begin{lstlisting}[language=json, caption={TravelBrain.postman\_collection.json (fragment)}, escapeinside={(*@}{@*)}]
    {
    "info": {
        "name": "TravelBrain API Tests",
        "schema": "https://schema.getpostman.com/json/collection/v2.1.0/"
    },
    "item": [
        {
        "name": "Authentication",
        "item": [
            {
            "name": "TC-AUTH-003: Register with Invalid Email",
            "request": {
                "method": "POST",
                "url": "(*@\\{\\{@*)base_url(*@\\}\\}@*)/api/auth/register",
                "body": {
                "mode": "raw",
                "raw": "{\n  \"email\": \"invalid-email\",\n  \"password\": \"Test123!\",\n  \"username\": \"test\"\n}"
                }
            },
            "event": [
                {
                "listen": "test",
                "script": {
                    "exec": [
                    "pm.test(\"Status is 400\", () => {",
                    "  pm.response.to.have.status(400);",
                    "});"
                    ]
                }
                }
            ]
            }
        ]
        }
    ]
    }
    \end{lstlisting}

    \subsection{Anexo D: Configuración OWASP ZAP}

    \begin{lstlisting}[language=bash, caption={Comandos ZAP CLI}]
    # Iniciar ZAP daemon
    docker run -u zap -p 8081:8080 -d zaproxy/zap-stable \
    zap.sh -daemon -host 0.0.0.0 -port 8080 \
    -config api.key=travelbrain2026

    # Spider scan
    curl "http://127.0.0.1:8081/JSON/spider/action/scan/" \
    -d "url=http://localhost:3001" \
    -d "apikey=travelbrain2026"

    # Active scan
    curl "http://127.0.0.1:8081/JSON/ascan/action/scan/" \
    -d "url=http://localhost:3001" \
    -d "recurse=true" \
    -d "apikey=travelbrain2026"

    # Generate HTML report
    curl "http://127.0.0.1:8081/OTHER/core/other/htmlreport/" \
    -d "apikey=travelbrain2026" > zap-report.html
    \end{lstlisting}

    \end{document}

